%%%%%%%%%%%%%%%%%%%%%%%%%%%%%%%%%%%%%%%%%%%%%%%%%%%%%%%%%%%%%%%%%%%%%%%%%%%%%%%%
\chapter{ПОСТАНОВКА ЗАДАЧИ, ВЫБОР СПОСОБА РЕШЕНИЯ}
%%%%%%%%%%%%%%%%%%%%%%%%%%%%%%%%%%%%%%%%%%%%%%%%%%%%%%%%%%%%%%%%%%%%%%%%%%%%%%%%

В современном мире коммуникации через интернет очень удобны. Поэтому областью разработки являются область организации VoIP-телефонии в web-приложениях. Однако VoIP-телефония особенно будет экономить время, если её встроить в web-приложение бизнес-системы. В частности звонки наиболее важны для CRM-систем.

В CRM-системах звонки занимают одну из центральных ролей для успешного бизнеса. У каждой CRM-системы имеется модуль Звонки, который для каждого звонка создаёт карточку. В ней можно записывать информацию о звонке, определить его значение (сделка, покупка, продажа и т.п.). Такие модули Звонков могут хранить историю о звонках, на основе которой будет составляться статистика, которая полезна управляющему персоналу компании. Для них важно знать сколько сделок совершили операторы, кто из них лучший. Функция записи звонков позволит выявлять плохих и хороших работников.

В конце раздела \ref{section:review} приведены пробелы организации VoIP-телефонии в браузере, которые есть в бизнес-системах и CRM-системах. Поэтому постараемся заполнить эти пробелы разработкой модуля VoIP-телефонии для web-браузера и внедрим его в одну из CRM-систем.

\section{Постановка задачи}

Итак, необходимо разработать модуль VoIP-телефонии для web-браузера. Данный модуль будет выполнять функцию телефона (софт-фона), который будет доступен прямо со страницы web-приложения. Так же данный модуль будет решением с отрытым исходным кодом. Это позволит тем кто его использует настраивать его под свои задачи.

Установим следующие требования к модулю:
\begin{enumerate}
\item Поддержка современными web-браузерами.
\item Модуль должен легко встраиваться в web-приложение.
\item Модуль должен быть расширяем (чтобы его встроить в конкретное web-приложение, достаточно было бы добавить только один компонент).
\item Модуль должен полностью контролировать состояние звонка.
\end{enumerate}

Обычно софт-фоны поддерживают передачу видео, аудио и мгновенных сообщений. Однако в CRM-системах видео-звонки практически не требуется. Поэтому в данной работе ограничимся передачей аудио потока.

Графический интерфейс должен состоять из скользящей кнопки и плавающего окна, которое появляется и скрывается по нажатию на эту кнопку. Плавающее окно должно включать поле для набора номера и кнопки <<вызов>> и <<сброс>>, которые должны меняются во время входящего звонка на кнопки <<снять трубку>> и <<отклонить звонок>> соответственно. То есть графический интерфейс должен реагировать на состояние звонка. Приведём ещё один пример. Графический интерфейс должен запрещать пользователю нажимать на кнопку сброса пока звонок ещё не начался.

\section{Выбор способа решения}

Как было рассмотренно в разделе \ref{subsection:reviewWebRTC} лучшей на сегодня технологией для организации VoIP-телефона в браузере является WebRTС. Данная технология встроенна в браузер и доступ к её компонентам имеется только с помощью JavaScript, который выполняется на web-странице. Поэтому разработка клиентской части будет реализованна на JavaScript.

Для обмена в VoIP-телефонии существуют два протокола H323 и SIP. По технологии WebRTC имеются только две реализации SIP протокола --- это библиотеки sipML5 и JsSIP. Реализации протокола H323 на сегодняшний день нет, поэтому этот протокол мы рассматривать не будем.

\subsection{Сравнительный анализ библиотек sipML5 и JsSIP}

Функциональные возможности библиотек sipML5 и JsSIP представленны в таблице \ref{table:sipml5_vs_jssip}.\cite{sipML5}\cite{JsSIP}

\begin{table}
	\caption{Функциональные возможности библиотек sipML5 и JsSIP}
	\begin{center}
	\begin{tabular}{|l|c|c|}
	\hline 
	\textbf{Функциональная возможность} & \textbf{sipML5} & \textbf{JsSIP}	\\ 
	\hline 
	аудио звонки & + & + \\ 
	\hline 
	видео звонки & + & + \\ 
	\hline 
	мгновенные сообщения & + & + \\ 
	\hline 
	статус присутствия в сети & + & + \\ 
	\hline 
	удержание вызова & + & + \\ 
	\hline 
	отключение микрофона & + & + \\ 
	\hline 
	тональный набор (DTMF) & + & + \\ 
	\hline 
	трансляция экрана & только для Chrome & - \\ 
	\hline 
	групповой звонок & + & - \\ 
	\hline 
	перенаправление вызова & + & - \\ 
	\hline 
	\end{tabular} 
	\end{center}
	\label{table:sipml5_vs_jssip}
\end{table}

Обе библиотеки имеют хорошую документацию. Размер библиотеки JsSIP ~130kb. Размер библиотеки sipML5 чуть больше 1Mb. Однако JsSIP требует установки Node.js на сервере.

Библиотека sipML5 работает на Chrome, Firefox, IE, Safari, Opera и Bowser. На каких браузерах работает библиотека JsSIP на сайте библиотеки не сказанно.

Обе библиотеки имеют свои преимущества, но для нашей задачи выберем библиотеку sipML5, потому что функциональных возможностей у неё больше и не нужно дополнительно устанавливать Node.js на сервер.

Ещё одним ограничением будет то, что WebRTC поддерживают не все браузеры, и библиотека sipML5 также работает не на всех браузерах.

\section{Резюме}

В данном разделе была поставлена задача разработки модуля SIP-телефонии для web-браузера, определены основные требования. Для разработки была выбрана технология WebRTC и библиотека sipML5 для осуществления звонков по протоколу SIP.