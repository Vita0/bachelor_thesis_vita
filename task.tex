%%%%%%%%%%%%%%%%%%%%%%%%%%%%%%%%%%%%%%%%%%%%%%%%%%%%%%%%%%%%%%%%%%%%%%%%%%%%%%%%
\chapter{ПОСТАНОВКА ЗАДАЧИ, ВЫБОР СПОСОБА РЕШЕНИЯ}
%%%%%%%%%%%%%%%%%%%%%%%%%%%%%%%%%%%%%%%%%%%%%%%%%%%%%%%%%%%%%%%%%%%%%%%%%%%%%%%%
В современном мире коммуникации через интернет становятся очень удобными, такие коммуникации экономят время. Такие коммуникации особенно будут экономить время, если их встраивать в CRM-системы. Поэтому задача будет следующая. Написать модуль SIP-телефона для веб-браузера, поддерживаемого как можно большим количеством браузеров, и легко в будущем встраиваемый в любой сайт, например, CRM-систему. В данной работе ограничимся только передачей аудио потока.

Для реализации данной задачи, будем использовать технологию WebRTC. Существуют две реализации технологии WebRTC по протоколу SIP на JavaScript. Это библиотеки sipML5 и JsSIP.

Преимущества sipML5:
\begin{enumerate}
\item наличие документации
\item поддержка трансфера звонка
\end{enumerate}

Недостатки sipML5:
\begin{enumerate}
\item большой размер 1 Mb
\item документация не очень подробная
\end{enumerate}

Преимущества JsSIP:
\begin{enumerate}
\item легковесная (~130kb)
\item подробная документация
\end{enumerate}

Недостатки JsSIP:
\begin{enumerate}
\item нет поддержки трансфера звонка
\item необходима установка NodeJS
\end{enumerate}
