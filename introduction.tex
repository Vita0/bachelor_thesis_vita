%%%%%%%%%%%%%%%%%%%%%%%%%%%%%%%%%%%%%%%%%%%%%%%%%%%%%%%%%%%%%%%%%%%%%%%%%%%%%%%%
\intro
%%%%%%%%%%%%%%%%%%%%%%%%%%%%%%%%%%%%%%%%%%%%%%%%%%%%%%%%%%%%%%%%%%%%%%%%%%%%%%%%
Определенной вехой в истории телекоммуникаций является VoIP-телефония (Voice over IP, VoIP), позволившая передавать "голос" по IP-сетям. По сравнению с аналоговой телефонией VoIP позволяет сократить расходы на разговоры, обеспечивает мобильность пользователей, а также позволяет передавать видео. В связи с этим IP-телефония получила большое распространение.

Многие бизнес-системы для звонков используют VoIP. Разработчикам бизнес-систем просто записывать информацию о таких звонках: имена собеседников, время начала и завершения звонка, длительность, а также сам звонок в аудио-файле. Такая информация полезна менеджерам, которые могут анализировать работу операторов.

Бизнес-система обычно являются web-приложением, напоминающем "социальную сеть компании". В таких случаях пользовательский интерфейс бизнес-системы в полном объёме доступен только через web-браузер. Поэтому телефония из браузера стала неотъемлемой частью бизнес-систем. Однако организовать VoIP-телефонию в браузере является непростой задачей.

Большинство бизнес-систем использует виртуальные серверы VoIP-телефонии. Тогда браузерные VoIP-телефоны реализованы под конкретный сервер VoIP-телефонии и приходится платить за виртуальный сервер.

Было бы гораздо удобнее, если был модуль VoIP-телефонии для web-браузера, который можно подключить к любой бизнес-системе.
