%%%%%%%%%%%%%%%%%%%%%%%%%%%%%%%%%%%%%%%%%%%%%%%%%%%%%%%%%%%%%%%%%%%%%%%%%%%%%%%%
\intro
%%%%%%%%%%%%%%%%%%%%%%%%%%%%%%%%%%%%%%%%%%%%%%%%%%%%%%%%%%%%%%%%%%%%%%%%%%%%%%%%
В последнее время очень популярными стали web-приложения, например социальные сети, игры, онлайн-редакторы (документов, изображений и видео), прямые видео-трансляции и многие-многие другие.

Около 15 лет назад клиент просматривал web-страницы, только переходя с одной на другую. Примерно в 2005 году, появился способ сделать страницы динамичными с помощью AJAX. С тех пор, почти весь обмен по HTTP инициировался клиентом разными способами, например каким-нибудь действием, или периодическим опросом сервера на получение новых данных. Однако при таком обмене появляется задержка на установление HTTP-соединения каждый раз при получении новых данных от сервера. Это создавало проблемы для создания web-приложений реального времени.(http://www.html5rocks.com/en/tutorials/websockets/basics/)

Около 5 лет назад появилась новая технология, которая позволила обмениваться двум сторонам асинхронно и симметрично. Это полнодуплексный протокол WebSocket, который работает поверх TCP. Уже в 2009 году вышла первая версия браузера, поддерживающая стандарт. (http://blog.chromium.org/2009/12/web-sockets-now-available-in-google.html)

//TODO