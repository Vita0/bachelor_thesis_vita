%%%%%%%%%%%%%%%%%%%%%%%%%%%%%%%%%%%%%%%%%%%%%%%%%%%%%%%%%%%%%%%%%%%%%%%%%%%%%%%%
\chapter{ТЕСТИРОВАНИЕ, АНАЛИЗ ПОЛУЧЕННЫХ РЕЗУЛЬТАТОВ}
%%%%%%%%%%%%%%%%%%%%%%%%%%%%%%%%%%%%%%%%%%%%%%%%%%%%%%%%%%%%%%%%%%%%%%%%%%%%%%%%

\section{Тестовое окружение}

На сервере установлена операционная система Debian GNU/Linux 8.4 (jessie). На сервере установлен сервер IP-телефонии Asterisk 13.7.2. В конфигурационных файлах настроены звонки с номерами формата XXX, и добавлены внутренние номера, позволяющие звонить при помощи web-сокетов по порту 8088. Для этого создан DTLS-сертификат, и включена поддержка icesupport.\cite{asterisk} Также добавлены номера, позволяющие звонить при помощи обычных сокетов по порту 5060.

На этом же сервере установлена CRM-система SalesPlatform vtiger CRM 6.4, которая работает на сервере Apache 2.4.10, в связке с PHP 5.6.22-0+deb8u1 и MySQL Server 5.5.49-0+deb8u1. Файлы разрабатываемого модуля помещены в корень CRM-системы. Чтобы подключить JavaScript файл к данной CRM-системе, необходимо добавить в базу данных CRM-системы в таблицу vtiger\_links запись c полями: linkurl - название файла (в нашем случае softPhone.js), linktype - HEADERSCRIPT, tabid - 0.\cite{vtiger_db} В этом случае скрипт будет выполнятся при каждой загрузке страницы.

