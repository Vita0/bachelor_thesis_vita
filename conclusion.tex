%%%%%%%%%%%%%%%%%%%%%%%%%%%%%%%%%%%%%%%%%%%%%%%%%%%%%%%%%%%%%%%%%%%%%%%%%%%%%%%%
\conclusion
%%%%%%%%%%%%%%%%%%%%%%%%%%%%%%%%%%%%%%%%%%%%%%%%%%%%%%%%%%%%%%%%%%%%%%%%%%%%%%%%
Пока что главной проблемой технологии WebRTC является её новизна. Пока что не удалось написать полностью кросс-браузерный код разработанного софт-фона. Было потрачено огромное количество времени на то, чтобы разобраться как использовать sipML5 для звонков. Документация оказалась менее подробной чем ожидалось. В связи с этим утерянным временем не удалось осуществить внедрение данного модуля в CRM-систему.

Однако основная часть модуля была написана. Звонки тестировались из браузера Firefox на сконфигурированном для WebRTC сервере IP-телефонии Asterisk. Остальные браузеры звонили, но с некоторыми неудачами. И это скорее всего вина не наша, так как демо-версия библиотеки sipML5 в тестируемых нами браузерах ведёт себя аналогично.

В итоге разработана alfa-версия модуля, встраиваемого в CRM-систему, который нужно доделывать. А именно необходимо реализовать реализовать серверную часть, которая получает информацию о SIP-аккаунте; регистрацию звонков в CRM-системе; доделать ядро модуля так, чтобы звонки осуществлялись в почти любом современном браузере.