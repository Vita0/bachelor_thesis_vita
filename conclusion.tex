%%%%%%%%%%%%%%%%%%%%%%%%%%%%%%%%%%%%%%%%%%%%%%%%%%%%%%%%%%%%%%%%%%%%%%%%%%%%%%%%
\conclusion
%%%%%%%%%%%%%%%%%%%%%%%%%%%%%%%%%%%%%%%%%%%%%%%%%%%%%%%%%%%%%%%%%%%%%%%%%%%%%%%%
Проведён анализ предметной области. Поставлена задача разработки модуля SIP-телефонии для web-браузера, решение которой заполняет пробелы VoIP-телефонии в бизнес-системах и CRM-системах. Для решения данной задачи выбрана современная технология WebRTC.

Основная часть модуля SIP-телефонии для web-браузера была реализована. Звонки тестировались из браузера Firefox на сконфигурированном для WebRTC сервере IP-телефонии Asterisk. Остальные браузеры звонили, но с некоторыми неудачами. Это скорее всего связанно с тем, что у технологии WebRTC отсутствует окончательный стандарт. RFC этой технологии находится в черновом варианте.

Также WebRTC гарантирует поддержку большинством браузеров, однако демо-версия библиотеки sipML5 в тестируемых нами браузерах ведёт себя с неудачами.

Большое количество времени было использовано на изучение библиотеки sipML5. Освоение её документации оказалось более сложной задачей, чем ожидалось. Поэтому на внедрение модуля SIP-телефонии в CRM-систему осталось меньше времени и данная задача была выполненна частично.

В итоге реализован модуль SIP-телефонии, который частично встроен в CRM-систему. Рассмотрим пути его развития.

Общие пути развития:
\begin{itemize}
\item Добавление поддержка всеми современными браузерами
\item Добавление поддержка большего количества бизнес-систем и CRM-систем
\item Автоматизированное тестирование
\end{itemize}

Пути развития для vtiger CRM:
\begin{itemize}
\item Реализация получения информации о SIP-аккаунте
\item Добавление регистрации звонков в CRM-систему
\end{itemize}
